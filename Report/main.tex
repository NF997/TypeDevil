
%%%%%%%%%%%%%%%%%%%%%%% file typeinst.tex %%%%%%%%%%%%%%%%%%%%%%%%%
%
% This is the LaTeX source for the instructions to authors using
% the LaTeX document class 'llncs.cls' for contributions to
% the Lecture Notes in Computer Sciences series.
% http://www.springer.com/lncs       Springer Heidelberg 2006/05/04
%
% It may be used as a template for your own input - copy it
% to a new file with a new name and use it as the basis
% for your article.
%
% NB: the document class 'llncs' has its own and detailed documentation, see
% ftp://ftp.springer.de/data/pubftp/pub/tex/latex/llncs/latex2e/llncsdoc.pdf
%
%%%%%%%%%%%%%%%%%%%%%%%%%%%%%%%%%%%%%%%%%%%%%%%%%%%%%%%%%%%%%%%%%%%


\documentclass[runningheads,a4paper]{llncs}

\usepackage{amssymb}
\setcounter{tocdepth}{3}
\usepackage{graphicx}
\usepackage{hyperref}
\usepackage{url}
\usepackage{listings}
\usepackage{color}
\definecolor{lightgray}{rgb}{.9,.9,.9}
\definecolor{darkgray}{rgb}{.4,.4,.4}
\definecolor{purple}{rgb}{0.65, 0.12, 0.82}

\lstdefinelanguage{JavaScript}{
  keywords={typeof, new, true, false, catch, function, return, null, catch, switch, var, if, in, while, do, else, case, break},
  keywordstyle=\color{blue}\bfseries,
  ndkeywords={class, export, boolean, throw, implements, import, this},
  ndkeywordstyle=\color{darkgray}\bfseries,
  identifierstyle=\color{black},
  sensitive=false,
  comment=[l]{//},
  morecomment=[s]{/*}{*/},
  commentstyle=\color{purple}\ttfamily,
  stringstyle=\color{red}\ttfamily,
  morestring=[b]',
  morestring=[b]"
}

\lstset{
   language=JavaScript,
   backgroundcolor=\color{lightgray},
   extendedchars=true,
   basicstyle=\footnotesize\ttfamily,
   showstringspaces=false,
   showspaces=false,
   numbers=left,
   numberstyle=\footnotesize,
   numbersep=9pt,
   tabsize=2,
   breaklines=true,
   showtabs=false,
   captionpos=b
}

\newcommand{\keywords}[1]{\par\addvspace\baselineskip

\noindent\keywordname\enspace\ignorespaces#1}

\begin{document}

\mainmatter  % start of an individual contribution

% first the title is needed
\title{TypeDevil: Dynamic Type Inconsistency Analysis for JavaScript\\-\\Seminar Report}

% a short form should be given in case it is too long for the running head
\titlerunning{TypeDevil - Seminar Report}

% the name(s) of the author(s) follow(s) next
\author{Nico Fechtner}
%
\authorrunning{Seminar Report: TypeDevil}
% (feature abused for this document to repeat the title also on left hand pages)

% the affiliations are given next; don't give your e-mail address
% unless you accept that it will be published
\institute{Technical University of Munich\\
Department of Informatics\\
Chair for IT Security\\
Boltzmannstraße 3, 85748 Garching, Germany\\
\href{mailto:nico.fechtner@tum.de}{nico.fechtner@tum.de}
}

%
% NB: a more complex sample for affiliations and the mapping to the
% corresponding authors can be found in the file "llncs.dem"
% (search for the string "\mainmatter" where a contribution starts).
% "llncs.dem" accompanies the document class "llncs.cls".
%

% ???
\toctitle{}
\tocauthor{}
\maketitle
% \newpage

\tableofcontents
\newpage

\begin{abstract}
JavaScript is dynamically and weakly typed which makes it possible to write type inconsistent code. 
This can often lead to bugs. 
TypeDevil addresses this issue with a mostly dynamic type inconsistency analysis which is able to warn developers about critical type related bugs.
An alternative approach would be to use an additional static type system like e.g. TypeScript.
\keywords{Dynamic Type Inconsistency Analysis for JavaScript, TypeDevil, Static Type Systems for JavaScript, TypeScript}
\end{abstract}


\section{Introduction}

This report is part of the seminar "Common Security Flaws in JavaScript based Applications", which was organized by Paul Muntean from the Chair of IT Security at the Faculty of Informatics of the Technical University of Munich.
The seminar took place in the summer semester 2018 and dealt with multiple scientific papers related to JavaScript Security. \\
I personally took a deeper look at the paper "TypeDevil: Dynamic Type Inconsistency Analysis for JavaScript", published in 2014 by Michael Pradel, Parker Schuh and Koushik Sen.
In this report I would like to explain the general problem the paper tries to address, introduce TypeDevil as a possible solution and compare TypeDevil to an alternative approach, namely additional static type systems. 

\section{Inconsistent Types as the Root Cause of Many Bugs}
First of all, I would like to introduce the general problem, TypeDevil wants to solve. \\
JavaScript has two interesting characteristics we will focus on. 
On the one hand, JavaScript is dynamically typed. This basically means, that we do not provide any static type annotations to our source code, like you would in statically typed languages, such as e.g. Java, and that types can change during runtime.
On the other hand, JavaScript is also weakly typed and thereby very permissive. In order to prevent runtime exceptions, JavaScript performs a lot of automatic type conversions, also known as implicit coercions.
Consider the code example. One could expect the output "Cat Dog Rabbit ".
The actual output however is "undefinedCat Dog Rabbit ". In the first iteration of the for-loop, we try to concatenate a string to the value of outputString, which is undefined. 
JavaScript now performs a implicit coercion and converts the value undefined to the string "undefined".

\lstset{language=javascript}
\begin{minipage}{\linewidth}
\begin{lstlisting}[frame=single, caption=Implicit Coercions]
var pets = ["Cat", "Dog", "Rabbit"];

var outputString;

for (var i in pets) {
    outputString += pets[i] + " ";
}

console.log(outputString);
\end{lstlisting}
\end{minipage}

As we saw, dynamic languages do not require programmers to annotate their programs with type information or to follow any strict typing discipline. 
This freedom allows programmers to write concise code in short time. 
However, most code does follow implicit type rules, e.g. only a single type per variable or object property or fixed function signatures.
The authors of TypeDevil also state that many bugs are actually violations of these rules.
So the freedom offered by dynamic languages often comes at the cost of hidden bugs. Since the language does not enforce any typing discipline, no compile-time warnings are reported if a program uses and combines types inconsistently.
Although the code example seems trivial, when you consider critical applications like authentication or payment libraries, of course even a little type error could do some real damage.

\section{TypeDevils Approach}

\subsection{Overview}

\subsection{Gathering Type Observations}

\subsection{Building the Type Graph and Identifying Inconsistent Types}

\subsection{Merging and Pruning of Warnings}

The online version of the volume will be available in LNCS Online.
Members of institutes subscribing to the Lecture Notes in Computer
Science series have access to all the pdfs of all the online
publications. Non-subscribers can only read as far as the abstracts. If
they try to go beyond this point, they are automatically asked, whether
they would like to order the pdf, and are given instructions as to how
to do so.

Please note that, if your email address is given in your paper,
it will also be included in the meta data of the online version.

\section{Related Research}



The correct BibTeX entries for the Lecture Notes in Computer Science
volumes can be found at the following Website shortly after the
publication of the book:
\url{http://www.informatik.uni-trier.de/~ley/db/journals/lncs.html}

\subsubsection*{Acknowledgments.} The heading should be treated as a
subsubsection heading and should not be assigned a number.

\section{Static Type Systems as an Alternative to Dynamic Analysis}

\section{Evaluation}

\subsection{Original Results for TypeDevil}

\subsection{Comparison between TypeDevil and TypeScript}

\subsubsection{Introduction to TypeScript}

\subsubsection{Evaluation Setup}

\subsubsection{Results}


\subsection{Discussion}


In order to permit cross referencing within LNCS-Online, and eventually
between different publishers and their online databases, LNCS will,
from now on, be standardizing the format of the references. This new
feature will increase the visibility of publications and facilitate
academic research considerably. Please base your references on the
examples below. References that don't adhere to this style will be
reformatted by Springer. You should therefore check your references
thoroughly when you receive the final pdf of your paper.
The reference section must be complete. You may not omit references.
Instructions as to where to find a fuller version of the references are
not permissible.

We only accept references written using the latin alphabet. If the title
of the book you are referring to is in Russian or Chinese, then please write
(in Russian) or (in Chinese) at the end of the transcript or translation
of the title.

Please note that proceedings published in LNCS are not cited with their
full titles, but with their acronyms!

\begin{thebibliography}{4}

\bibitem{jour} Smith, T.F., Waterman, M.S.: Identification of Common Molecular
Subsequences. J. Mol. Biol. 147, 195--197 (1981)

\bibitem{lncschap} May, P., Ehrlich, H.C., Steinke, T.: ZIB Structure Prediction Pipeline:
Composing a Complex Biological Workflow through Web Services. In: Nagel,
W.E., Walter, W.V., Lehner, W. (eds.) Euro-Par 2006. LNCS, vol. 4128,
pp. 1148--1158. Springer, Heidelberg (2006)

\bibitem{book} Foster, I., Kesselman, C.: The Grid: Blueprint for a New Computing
Infrastructure. Morgan Kaufmann, San Francisco (1999)

\bibitem{proceeding1} Czajkowski, K., Fitzgerald, S., Foster, I., Kesselman, C.: Grid
Information Services for Distributed Resource Sharing. In: 10th IEEE
International Symposium on High Performance Distributed Computing, pp.
181--184. IEEE Press, New York (2001)

\bibitem{proceeding2} Foster, I., Kesselman, C., Nick, J., Tuecke, S.: The Physiology of the
Grid: an Open Grid Services Architecture for Distributed Systems
Integration. Technical report, Global Grid Forum (2002)

\bibitem{url} National Center for Biotechnology Information, \url{http://www.ncbi.nlm.nih.gov}

\end{thebibliography}


\section*{Appendix: Springer-Author Discount}

LNCS authors are entitled to a 33.3\% discount off all Springer
publications. Before placing an order, the author should send an email, 
giving full details of his or her Springer publication,
to \url{orders-HD-individuals@springer.com} to obtain a so-called token. This token is a
number, which must be entered when placing an order via the Internet, in
order to obtain the discount.

\section{Conclusion and Future Work}
Here is a checklist of everything the volume editor requires from you:


\begin{itemize}
\settowidth{\leftmargin}{{\Large$\square$}}\advance\leftmargin\labelsep
\itemsep8pt\relax
\renewcommand\labelitemi{{\lower1.5pt\hbox{\Large$\square$}}}

\item The final \LaTeX{} source files
\item A final PDF file
\item A copyright form, signed by one author on behalf of all of the
authors of the paper.
\item A readme giving the name and email address of the
corresponding author.
\end{itemize}
\end{document}
